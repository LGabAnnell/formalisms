\documentclass{article}
\usepackage[utf8]{inputenc}
\usepackage[english]{babel}
\usepackage{lipsum}
\usepackage{listings}
\usepackage{setspace}
\usepackage{fancyhdr}
\usepackage{graphicx}
\usepackage{wrapfig}
\usepackage[margin=3cm, tmargin=2cm, bmargin=2cm]{geometry}
\usepackage{geometry}
\usepackage{changepage}
\usepackage{url}
\usepackage{gensymb}
\usepackage{caption}
\usepackage{subcaption}
\expandafter\def\csname ver@subfig.sty\endcsname{}
\usepackage{subfig}
\usepackage{minted}
\usepackage[hidelinks]{hyperref}
\usepackage{syntax}
\setminted[lua]{tabsize=2}

\setlength{\grammarindent}{0cm}

\pagestyle{fancy}
\fancyhf{}
\fancyhead[l]{\small Lars Gabriel \linebreak Annell Rydenvald}
\fancyhead[c]{
	UNIGE \linebreak
}
\fancyhead[R]{Faculté des Sciences \linebreak Département d'informatique}
\fancyfoot[C]{
	\thepage
}

\lstset{
	basicstyle=\ttfamily
}

\captionsetup[subfigure]{labelformat=empty}


\begin{document}
\begin{center}
	\huge 
	Ardoises \\
	Formalismes d'opérateurs et expressions
\end{center}
\renewcommand{\contentsname}{Sommaire}
\tableofcontents
\newpage

\section{Introduction}
Dans le cadre du logiciel \underline{\href{https://ardoises.ovh/overview}{Ardoises}}, ce travail a pour but de fournir un outil de génération de grammaire et d'analyseur lexical, afin de donner aux utilisateurs la possibilité de définir des formalismes opérateurs. L'objectif de l'analyseur lexical est de produire un arbre syntaxique abstrait à partir des opérateurs définis par l'utilisateur, ainsi que de vérifier les propriétés qui leur sont données. % ce travail a pour but d'introduire la possibilité aux utilisateurs de définir des formalismes opérateurs, de fournir un analyseur lexical capable de lire une expression basée sur ces opérateurs et d'en produire un arbre syntaxique abstrait, ainsi que de vérifier que les propriétés des opérateurs définis par les utilisateurs soient respectées.
\section{Ardoises}
% Ecrire à propos d'ardoises, et des autres logiciels qui existent déjà (c.f. présentation)
\section{Formalismes d'opérateurs}
% Formalismes des opérateurs
\section{Analyse lexicale}
Deux solutions pour effectuer l'analyse lexicale se présentaient à nous:
\begin{enumerate}
	\item Utiliser des expressions régulières, qui seraient définies par l'utilisateur pour chaque opérateur.
	\item Générer une grammaire à partir de modèles (patterns) prédéfinis pour différents types d'opérateurs.
\end{enumerate}
J'ai choisi d'utiliser la 2$^{\grave{e}me}$ méthode, car même si les expressions régulières auraient pû être un moyen "plus simple", elles posent un problème: les expressions arithmétiques incluant des parenthèses ne sont pas des langages réguliers, puisqu'on peut, en prenant un exemple simple, se retrouver dans un cas comme celui-ci: $\{(\textquoteleft(\textquoteleft^{n}\textquoteleft)\textquoteleft^{n})^{*}, n \ge 0 \}$
\subsection{Choix de l'outil}
Il existe plusieurs générateurs d'analyseurs lexicaux, tels que: \begin{enumerate}
	\item \underline{\href{http://www.antlr.org/about.html}{ANTLR}}, écrit en Java, générant un analyseur $LL(^*)$
	\item \underline{\href{http://dinosaur.compilertools.net/}{Flex/Bison}}, écrit en C, générant un analyseur $LALR(1)$
	\item \underline{\href{https://pegjs.org/}{PEGjs}}, écrit en JavaScript, générant un parser $PEG$
	\item \underline{\href{https://github.com/LuaDist/lpeg}{LPeg}}, écrit en C, compatible avec Lua, générant un parser $PEG$
	\item \underline{\href{https://github.com/pygy/LuLPeg}{LulPeg}}, implémentation en pur Lua de $LPeg$.
\end{enumerate}
Puisque le projet est écrit en Lua, pour avoir une interopérabilité facile entre les définitions d'opérateurs et expressions, j'ai choisi d'utiliser LPeg. LulPeg a aussi été considéré, cependant celui-ci est beaucoup plus lent que LPeg. \\ \textit{TODO: Include justification.} \\ LPeg, par sa modularité, permet aussi de facilement générer une grammaire à partir des opérateurs définis par l'utilisateur.
\subsection{Algorithme de génération de grammaire}
Puisque nous ne connaissons pas en avance les opérateurs disponibles dans une expression, il est nécessaire de générer la grammaire à partir des opérateurs définis par l'utilisateur. \\
Supposons les opérateurs simples suivants:
\begin{center}
\begin{tabular}{|c|c|c|c|c|}
	\hline
	\textbf{Opérateur} & \textbf{Représentation} & \textbf{Priorité} &  \textbf{Type} & \textbf{Associativité}\\ \hline
	Addition & '+' & 11 & binaire & droite \\ \hline
	Soustraction & '-' & 11 & binaire & droite  \\ \hline
	Multiplication & '*' & 12 & binaire & droite  \\ \hline
	Division & '/' & 12 & binaire & droite  \\ \hline
	Exponentiation & '\textasciicircum' & 13 & binaire & droite  \\ \hline
	Négation    & '-'  & 14 & unaire préfixe & N/A  \\ \hline 
  nombre & [0-9]+ & 15 & litéral & N/A \\ \hline
\end{tabular}
\end{center}
Prenons comme base la simple grammaire suivante (qui ne fait rien):
\begin{grammar}
	<s> ::= <s'>
\end{grammar}
On va ensuite rajouter un à un les opérateurs dans cette grammaire, de la manière  suivante:

\begin{enumerate}
	\item On trie la table des opérateurs, par priorité, dans l'ordre décroissant.
	\item Pour chaque type d'opérateur, on crée le pattern correspondant. Par exemple, l'opérateur binaire $Addition$ dans notre table, aura le pattern
	\begin{grammar}
		<Addition> ::= <P12> `+' ( <Addition> | <P12> )
	\end{grammar}
	\item On rajoute le nouveau pattern crée dans la grammaire de base
\end{enumerate}
Pour finir, on se retrouve avec une grammaire comme celle-ci:
\begin{grammar}
<s> \ \ \ \ ::= <s'>\\
<s'> \ \ \ ::= <P11>\\
<P11> ::= <Addition> | <Soustraction> | <P12>\\
<P12> ::= <Division> | <Multiplication> | <P13>\\
<P13> ::= <Exponentiation> | <P14>\\
<P14> ::= <Négation> | <P15>\\
<P15> ::= <nombre> | `(' <s> `)'
\end{grammar}
Puisque $LPeg$ génère un analyseur lexical de type $PEG$, il faut faire attention à l'ordre de déclaration de la grammaire pour avoir une précédence correcte.

\subsection{Patterns utilisés pour les différents types d'opérateurs}
Les fonctions pour générer les patterns sont stockés dans une table, et retournent un pattern $LPeg$. Considérons ici que les espaces blancs (\textbackslash s, \textbackslash t, \textbackslash n) sont ignorés. De plus, notons la "prochaine" expression dans la grammaire par $next\_expression$, et l'expression courante par $curr\_expression$, et la représentation de l'opérateur (e.g. "+") par $op\_repr$. Notons aussi l'axiome $s$ par $Expression$
\subsubsection{Opérateurs binaires}
Ce pattern est un des plus simples et un des plus évidents.
\begin{grammar}
	<binaire> ::= <next\_expression> <op\_repr> ( <curr\_expression> | <next\_expression> )
\end{grammar}
\subsubsection{Opérateurs ternaires}
Il existe deux variantes d'opérateurs ternaires, la première étant le bloc conditionnel, "if ... then ... else", et l'autre ressemblant au "inline if", par exemple "... ? ... : ...". Pour décrire ces deux opérateurs, on dénombre les opérateurs par $op\_reprN, \ N \in {1, 3}$.
\begin{enumerate}
	\item Pour la première version, le pattern utilisé est 
	\begin{grammar}
		<ternaire\_v1> ::= <op\_repr1> <Expression> <op\_repr2> <Expression> <op\_repr3> <Expression>
	\end{grammar}
	\item Pour la deuxième version, on utilisera
	\begin{grammar}
		<ternaire\_v2> ::= <next\_expression> <op\_repr1> <Expression> <op\_repr2> ( <curr\_expression> | <next\_expression> )
	\end{grammar}
\end{enumerate}
Il faut cependant faire attention à la priorité qu'on fixe dans les deux cas différents. Le premier cas requiert une priorité élevée (normalement au même niveau que les terminaux), et le deuxième cas nécessite une priorité basse.
\subsubsection{Opérateurs unaires préfixes}
Le pattern lui-même n'est pas difficile à définir:
\begin{grammar}
	<unaire\_préfixe> ::= <op\_repr> ( <curr\_expression> | <next\_expression> )
\end{grammar}
Cependant on est dans un cas où il faut aussi donner une priorité élevée à l'opérateur pour qu'il soit reconnu dans tous les cas.
\subsubsection{Opérateurs unaires postfixes}
Pour ce type d'opérateur, on est obligé de faire une répétition sur la représentation de l'opérateur, à cause du fait qu'on n'a pas la possibilité d'utiliser la récursion à gauche.
\begin{grammar}
	<unaire\_postfixe> ::= <next\_expression> <op\_repr>+
\end{grammar}
\subsubsection{Opérateurs n-aires}
Pour représenter les opérateurs n-aires, nous avons décidé d'utiliser la forme "représentation\_opérateur'('opérande1, opérande2, ..., opérandeN')', ce qui nous donne:
\begin{grammar}
	<naire> ::= <op\_repr> `(' <Expression> \{ `,' <Expression> \} `)'
\end{grammar}
\subsection{Associativité à gauche}
On constate que tous les opérateurs binaires dans le tableau précédent sont associatifs à droite. Or, on voudrait pouvoir préciser l'associativité lors de la déclaration des opérateurs, puisque cela modifie l'arbre syntaxique produit par l'analyseur lexical. Dans le cas d'un analyseur $LR$, l'associativité à gauche se traduit simplement par une récursion à gauche, telle que
\begin{grammar}
	<Expr> ::= <Expr> `+' <nombre> | <nombre>
\end{grammar}
Malheureusement, $LPeg$ ne peut pas gérer les récursions à gauche, et une élimination de la récursion à gauche du type
\begin{grammar}
	<Expr> \ ::= <nombre> <Expr'> \\
	<Expr'> ::= `+' <Expr'> | <nombre>
\end{grammar}
ne suffit pas. On a encore une associativité à droite. \\ \\
Il existe des implémentations pour introduire la récursion à gauche dans les grammaires $PEG$, telles que \underline{\href{https://github.com/norswap/whimsy/blob/master/doc/autumn/README.md}{Autumn}}\textsuperscript{[1]} ou celle proposée par Laurence Tratt\textsuperscript{[2]}. Puisque nous utilisons $LPeg$ et Lua, aucune de ces deux méthodes ne sont possibles dans le cadre de ce travail, il nous faut donc notre propre implémentation pour gérer les opérateurs binaires étant déclarés comme étant associatifs à gauche.
\subsubsection{Implémentation}
$LPeg$ nous permet de déclarer des fonctions de capture pour ce qui est lu par l'analyseur généré. Supposons la grammaire suivante:
\begin{grammar}
	<s> ::= <nombre> `+' ( <s> | <nombre> )
\end{grammar}
Notre fonction de capture pour le modèle qui suit $s$ recevra simplement les arguments $gauche$, \\ $repr\acute{e}sentation \ de \ l'op\acute{e}rateur$, $droite$. Déclarons d'abord la fonction comme ceci:
\begin{minted}{lua}
function left_assoc(left, op, right)
  return { left = left, op = op, right = right }
end
\end{minted}
Une analyse de l'entrée \mint{lua}{1 + 2 + 3} \noindent nous retournera donc la table \mint{lua}{{ left = 1, op = +, right = { left = 2, op = +, right = 3 } }}
\noindent Si on veut avoir l'associativité à gauche, ce résultat est évidemment faux. Pour avoir une production correcte, il nous suffit d'opérer de manière récursive. On a donc (en pseudo-code):
\begin{minted}{lua}
-- pattern est le pattern pour l'analyseur lexical
-- _op est l'opérateur auquel on veut appliquer l'associativité à gauche
function left_assoc(pattern, _op)
  -- left est la partie gauche de l'expression parsée
  -- op est l'opérateur
  -- right est la partie droite
  function recursion(left, op, right)
		-- Est-ce que nous avons une table à droite? Dans ce cas il faut
		-- "voler" l'opérande gauche de l'opérande de droite
    if right is a table and op equals _op.operator then
			-- Par la manière dont fonctionne LPeg, il faut quand même faire une
			-- récursion sur l'opérateur à gauche, si celui-ci est binaire et
			-- associatif à gauche
			if left is a table and the left operator is a binary left-associative operator then
				left = recursion(left.left, left.op, left.right)
			end
	    
	    -- Cependant il faut faire attention à ce que l'opérateur de droite
	    -- soit aussi un opérateur binaire, et que celui-ci ait une priorité
	    -- soit égale ou moins élevée que l'opérateur qu'on traite maintenant
	    if right.op.priority > _op.priority or right.op.type is not binary then
		    -- Dans ce cas on retourne normalement
	      return { left = left, op = op, right = right }
	    end
    
	    -- Sinon on fait la récursion
      return recursion(
        {
          left  = left,
          op    = op,
          right = right.left
        },
        right.op,
        right.right
      )
    else
	    -- Par la manière dont fonctionne LPeg, il faut quand même faire une
	    -- récursion sur l'opérateur à gauche, si celui-ci est binaire et
	    -- associatif à gauche
	    if left is a table and the left operator is a binary left-associative operator then
		    left = recursion(left.left, left.op, left.right)
	    end
			
	    -- On retourne normalement, ce que nous avons à droite n'est pas
	    -- une table, donc il n'y a pas besoin de traiter
      return { left = left, op = op, right = right }
    end
  end
  
  -- L'opérateur '/' est surchargé par LPeg, ceci retourne le pattern et la 
  -- la fonction "recursion" sera la fonction de capture lorsque ce pattern
  -- est rencontré
  -- http://www.inf.puc-rio.br/\~roberto/lpeg/#cap-func
  return pattern / recursion
end
\end{minted}
Pour revenir à notre expression \mint{lua}{1 + 2 + 3} \noindent de plus haut, si nous avons donné la propriété d'associativité à gauche à notre opérateur \lstinline|+|, l'analyse lexicale nous retournera maintenant la table correcte \begin{minted}{lua}
	{ left = { left = 1, op = "+", right = 2 }, op = "+", right = 3 }
\end{minted}
\subsection{Autres fonctions de capture}
\subsubsection{Opérateurs unaires postfixes}
L'impossibilité d'utiliser la récursion à gauche avec $LPeg$ pose aussi un problème pour les opérateurs postfixes. Si on avait la récursion à gauche, on aurait simplement pu écrire
\begin{grammar}
	<unaire\_postfixe> ::= ( <curr\_expression> | <next\_expression> ) <op\_repr>
\end{grammar}
Supposons l'opérateur postfixe $`\%`$.
Avec le pattern précédent on respecte la priorité de déclaration des opérateurs, et on peut aussi capturer une expression du type `3\%\%\%` qui nous donnerait cette production correcte \mintinline{lua}{{ { { 3 % } % } % }}.
\noindent Sans la récursion à gauche, on ne peut plus capturer l'entièreté de l'expression `3\%\%\%` sans utiliser de répétition. Or la répétition du symbole nous donne une table comme celle-ci:
\mintinline{lua}{{ 3 % % % }} ce qui n'est pas ce qu'on veut. On va donc opérer sur la liste qui nous est donnée par $LPeg$ pour créer la table juste.
\begin{minted}{lua}
local function postfix_capture(p)
	-- function to handle postfix operators when we receive them
	-- The pattern is defined as Exp * op_representation^1,
	-- which means that for an input such as "3~~" we would get
	-- a table such as { 3 ~ ~ }, whereas with this function we get
	-- { { 3 ~ } ~ }
	local function construct(list)
	  -- On sait que la table va ressembler à quelque chose 
	  -- comme { expression ~ ~ ~ ~ }. On commence donc par le dernier élément
	  -- de la table, et on 
		local function rec(n, t)
		  if n == 1 then
			  return list[1]
		  else
				t = { op = list[n], left = rec(n - 1, { t }) }
		  end
		  t.op_type = "unary_postfix"
		  return t
		end
	
	  return rec(tlen(list), { })
	end
	
	-- `...' est la liste des arguments
	-- (donne tous les arguments passés à la fonction)
	-- `{ ... }' nous donne simplement une table contenant les arguments
	-- passés à la fonction
	return p / function(...)
		return construct({ ... })
	end
end
\end{minted}
\subsubsection{Opérateurs n-aires}
Pour les opérateurs n-aires, la fonction de capture transforme simplement les expressions capturées qui sont en fait un terminal, par exemple "3" ou "variable" en une table.
\begin{minted}{lua}
local function nary_node (p)
	return p / function (op, ...)
		local arglist = { }
		-- select("#", ...) retourne le nombre d'arguments
		-- passés à la fonction
		local arglen  = select("#", ...)

		for i = 1, arglen, 1 do
			-- select(i, ...) retourne le i-ème argument
			local v = select(i, ...)
			if type(v) ~= "table" then
				arglist[i] = { v }
			else
				arglist[i] = v
			end
		end

		return { op = op, operands = arglist, op_type = "nary" }
	end
end
\end{minted}
\subsection{Créer une règle pour un nouveau type d'opérateur}
Pour permettre aux utilisateurs d'utiliser un type d'opérateur qui n'est pas déjà défini dans le générateur de grammaire, il faut suivre quelques étapes simples. Faisons un exemple, un opérateur quaternaire, étape par étape:
\begin{enumerate}
	\item Décider du nom du type d'opérateur. Utilisons $quaternary$
	\item Créer une entrée dans la table \lstinline|patterns| avec le même nom que le type d'opérateur, ce qui devrait ressembler à ceci:
	\begin{minted}{lua}
quaternary = function (operator, curr_expr, next_expr)

end
	\end{minted}
	\lstinline|operator| est l'objet opérateur passé, ce qui nous permet de récupérer ses propriétés, \lstinline|curr_expr| est une référence vers la règle elle-même, pour qu'on puisse faire la récursion à droite. Finalement, \lstinline|next_expr| est une référence vers la règle suivante dans la grammaire (pour la descente récursive).
	\item Créer la règle du type d'opérateur. Pour celui-ci, utilisons
	\begin{grammar}
		<quaternary> ::= <next\_expr> <op\_repr1> <Expression> <op\_repr2> <Expression> <op\_repr3> ( <quaternary> | <next\_expr> )
	\end{grammar}
	qui se traduit en Lua par
	\begin{minted}{lua}
local pattern = (
	next_expr * white * first * white *
	lp.V("axiom") * white * second * white * 
	lp.V("axiom") * white * third * white * (curr_expr + next_expr) 
)
	\end{minted}
	Quelques explications:
	\begin{enumerate}
		\item \mintinline{lua}{lp.V("axiom")} fait référence à l'axiome, c'est-à-dire qu'on autorise toutes les expressions à l'intérieur de l'opérateur.
		\item \mintinline{lua}{first}, \mintinline{lua}{second} et \mintinline{lua}{third} sont les représentations en caractère (ou chaîne de caractères) de l'opérateur. Ils peuvent être identiques.
		\item \mintinline{lua}{white} doit obligatoirement être utilisé pour ignorer les caractères blancs (\lstinline|\s|, \lstinline|\t|, \lstinline|\n|).
		\item Il faut utiliser \lstinline|next_expr| sur la gauche pour la descente récursive. Sur la droite de la règle, il faut utiliser \lstinline|curr_expr| pour la récursion à droite (pour pouvoir enchaîner des opérateurs qui ont la même priorité).
	\end{enumerate}
	\item Définir une fonction de capture. Dans le cas de notre opérateur quaternaire, il est simplement question de retourner une table contenant l'opérateur et les opérandes, puisqu'il n'y a pas de propriétés telles que l'associativité à gauche. \\
	Les fonctions de capture avec $LPeg$ reçoivent en argument ce qui est lu, et les arguments vont dans l'ordre, de gauche à droite. Donc, pour cet opérateur, on définirait la fonction de capture de cette manière:
	\begin{minted}{lua}
-- p est la règle
function quaternary_capture (p)
	-- On utilise l'opérateur de capture de LPeg
	return p / function (left, op1, middle1, op2, middle2, op3, right)
		-- Et on retourne juste une table.
		-- A savoir: on pourrait utiliser la fonction lpeg.Ct (Capture table),
		-- cependant il est préférable de définir nos clés nous-même, car lpeg.Ct
		-- retourne une table dont les clés sont des nombres (1, 2, ...)
		return {
			left    = left,
			op      = op1,
			op2     = op2,
			op3     = op3,
			middle1 = middle1,
			middle2 = middle2,
			right   = right,
			op_type = "quaternary"
		}
	end
end
	\end{minted}
	\item Finaliser le tout. L'entrée \lstinline|quaternary| dans notre table \lstinline|patterns| devrait ressembler à ceci:
	\begin{minted}{lua}
quaternary = function (operator, curr_expr, next_expr)
	-- On commence par récupérer les représentations textuelles de l'opérateur
	local first, second, third = lp.P(operator.operator), 
	lp.P(operator.operator1), lp.P(operator.operator2)
	
	-- On dit à LPeg de capturer ces représentations
	first = lp.C(first)
	second = lp.C(second)
	third = lp.C(third)
	
	local pattern = (
		next_expr * white * first * white *
		lp.V("axiom") * white * second * white * lp.V("axiom") * white *
		third * white * (curr_expr + next_expr)
	)
	
	-- Finalement on retourne la fonction de capture, 
	-- et notre règle sera insérée dans notre grammaire globale.
	return quaternary_capture(pattern)
end,
	\end{minted}
	\item En définissant notre opérateur de cette façon
	\begin{minted}{lua}
local quaternary = {
	operator  = "u",
	operator1 = "u",
	operator2 = "u",
	type      = "quaternary",
	priority  = 13
}
	\end{minted}
	on peut maintenant écrire \mint{lua}{1 u 2 u 3 u 4} ce qui produit la table
	\begin{minted}{lua}
{
	left    = "1",
	middle1 = "2",
	middle2 = "3",
	right   = "4",
	op      = "u",
	op2     = "u",
	op3     = "u",
	op_type = "quaternary"
}
	\end{minted}
\end{enumerate}
\newpage
\noindent[1]: Parsing Expression Grammars Made Practical, Nicolas Laurent \newline
[2]: Direct Left-Recursive Parsing Expression Grammars, Laurence Tratt

\end{document}